% Options for packages loaded elsewhere
% Options for packages loaded elsewhere
\PassOptionsToPackage{unicode}{hyperref}
\PassOptionsToPackage{hyphens}{url}
\PassOptionsToPackage{dvipsnames,svgnames,x11names}{xcolor}
%
\documentclass[
]{scrartcl}
\usepackage{xcolor}
\usepackage[left=2.5cm,right=2.5cm,top=3cm,bottom=3cm]{geometry}
\usepackage{amsmath,amssymb}
\setcounter{secnumdepth}{5}
\usepackage{iftex}
\ifPDFTeX
  \usepackage[T1]{fontenc}
  \usepackage[utf8]{inputenc}
  \usepackage{textcomp} % provide euro and other symbols
\else % if luatex or xetex
  \usepackage{unicode-math} % this also loads fontspec
  \defaultfontfeatures{Scale=MatchLowercase}
  \defaultfontfeatures[\rmfamily]{Ligatures=TeX,Scale=1}
\fi
\usepackage{lmodern}
\ifPDFTeX\else
  % xetex/luatex font selection
\fi
% Use upquote if available, for straight quotes in verbatim environments
\IfFileExists{upquote.sty}{\usepackage{upquote}}{}
\IfFileExists{microtype.sty}{% use microtype if available
  \usepackage[]{microtype}
  \UseMicrotypeSet[protrusion]{basicmath} % disable protrusion for tt fonts
}{}
\makeatletter
\@ifundefined{KOMAClassName}{% if non-KOMA class
  \IfFileExists{parskip.sty}{%
    \usepackage{parskip}
  }{% else
    \setlength{\parindent}{0pt}
    \setlength{\parskip}{6pt plus 2pt minus 1pt}}
}{% if KOMA class
  \KOMAoptions{parskip=half}}
\makeatother
% Make \paragraph and \subparagraph free-standing
\makeatletter
\ifx\paragraph\undefined\else
  \let\oldparagraph\paragraph
  \renewcommand{\paragraph}{
    \@ifstar
      \xxxParagraphStar
      \xxxParagraphNoStar
  }
  \newcommand{\xxxParagraphStar}[1]{\oldparagraph*{#1}\mbox{}}
  \newcommand{\xxxParagraphNoStar}[1]{\oldparagraph{#1}\mbox{}}
\fi
\ifx\subparagraph\undefined\else
  \let\oldsubparagraph\subparagraph
  \renewcommand{\subparagraph}{
    \@ifstar
      \xxxSubParagraphStar
      \xxxSubParagraphNoStar
  }
  \newcommand{\xxxSubParagraphStar}[1]{\oldsubparagraph*{#1}\mbox{}}
  \newcommand{\xxxSubParagraphNoStar}[1]{\oldsubparagraph{#1}\mbox{}}
\fi
\makeatother


\usepackage{longtable,booktabs,array}
\usepackage{calc} % for calculating minipage widths
% Correct order of tables after \paragraph or \subparagraph
\usepackage{etoolbox}
\makeatletter
\patchcmd\longtable{\par}{\if@noskipsec\mbox{}\fi\par}{}{}
\makeatother
% Allow footnotes in longtable head/foot
\IfFileExists{footnotehyper.sty}{\usepackage{footnotehyper}}{\usepackage{footnote}}
\makesavenoteenv{longtable}
\usepackage{graphicx}
\makeatletter
\newsavebox\pandoc@box
\newcommand*\pandocbounded[1]{% scales image to fit in text height/width
  \sbox\pandoc@box{#1}%
  \Gscale@div\@tempa{\textheight}{\dimexpr\ht\pandoc@box+\dp\pandoc@box\relax}%
  \Gscale@div\@tempb{\linewidth}{\wd\pandoc@box}%
  \ifdim\@tempb\p@<\@tempa\p@\let\@tempa\@tempb\fi% select the smaller of both
  \ifdim\@tempa\p@<\p@\scalebox{\@tempa}{\usebox\pandoc@box}%
  \else\usebox{\pandoc@box}%
  \fi%
}
% Set default figure placement to htbp
\def\fps@figure{htbp}
\makeatother


% definitions for citeproc citations
\NewDocumentCommand\citeproctext{}{}
\NewDocumentCommand\citeproc{mm}{%
  \begingroup\def\citeproctext{#2}\cite{#1}\endgroup}
\makeatletter
 % allow citations to break across lines
 \let\@cite@ofmt\@firstofone
 % avoid brackets around text for \cite:
 \def\@biblabel#1{}
 \def\@cite#1#2{{#1\if@tempswa , #2\fi}}
\makeatother
\newlength{\cslhangindent}
\setlength{\cslhangindent}{1.5em}
\newlength{\csllabelwidth}
\setlength{\csllabelwidth}{3em}
\newenvironment{CSLReferences}[2] % #1 hanging-indent, #2 entry-spacing
 {\begin{list}{}{%
  \setlength{\itemindent}{0pt}
  \setlength{\leftmargin}{0pt}
  \setlength{\parsep}{0pt}
  % turn on hanging indent if param 1 is 1
  \ifodd #1
   \setlength{\leftmargin}{\cslhangindent}
   \setlength{\itemindent}{-1\cslhangindent}
  \fi
  % set entry spacing
  \setlength{\itemsep}{#2\baselineskip}}}
 {\end{list}}
\usepackage{calc}
\newcommand{\CSLBlock}[1]{\hfill\break\parbox[t]{\linewidth}{\strut\ignorespaces#1\strut}}
\newcommand{\CSLLeftMargin}[1]{\parbox[t]{\csllabelwidth}{\strut#1\strut}}
\newcommand{\CSLRightInline}[1]{\parbox[t]{\linewidth - \csllabelwidth}{\strut#1\strut}}
\newcommand{\CSLIndent}[1]{\hspace{\cslhangindent}#1}



\setlength{\emergencystretch}{3em} % prevent overfull lines

\providecommand{\tightlist}{%
  \setlength{\itemsep}{0pt}\setlength{\parskip}{0pt}}



 


\usepackage[noblocks]{authblk}
\renewcommand*{\Authsep}{, }
\renewcommand*{\Authand}{, }
\renewcommand*{\Authands}{, }
\renewcommand\Affilfont{\small}
\usepackage{lineno}
\usepackage{scrlayer-scrpage}
\rohead{IOTC-2025-WPDCS21-}
\usepackage{lscape}
\newcommand{\blandscape}{\begin{landscape}}
\newcommand{\elandscape}{\end{landscape}}
\makeatletter
\@ifpackageloaded{caption}{}{\usepackage{caption}}
\AtBeginDocument{%
\ifdefined\contentsname
  \renewcommand*\contentsname{Table of contents}
\else
  \newcommand\contentsname{Table of contents}
\fi
\ifdefined\listfigurename
  \renewcommand*\listfigurename{List of Figures}
\else
  \newcommand\listfigurename{List of Figures}
\fi
\ifdefined\listtablename
  \renewcommand*\listtablename{List of Tables}
\else
  \newcommand\listtablename{List of Tables}
\fi
\ifdefined\figurename
  \renewcommand*\figurename{Figure}
\else
  \newcommand\figurename{Figure}
\fi
\ifdefined\tablename
  \renewcommand*\tablename{Table}
\else
  \newcommand\tablename{Table}
\fi
}
\@ifpackageloaded{float}{}{\usepackage{float}}
\floatstyle{ruled}
\@ifundefined{c@chapter}{\newfloat{codelisting}{h}{lop}}{\newfloat{codelisting}{h}{lop}[chapter]}
\floatname{codelisting}{Listing}
\newcommand*\listoflistings{\listof{codelisting}{List of Listings}}
\makeatother
\makeatletter
\makeatother
\makeatletter
\@ifpackageloaded{caption}{}{\usepackage{caption}}
\@ifpackageloaded{subcaption}{}{\usepackage{subcaption}}
\makeatother
\usepackage{bookmark}
\IfFileExists{xurl.sty}{\usepackage{xurl}}{} % add URL line breaks if available
\urlstyle{same}
\hypersetup{
  pdftitle={Impacts of the sampling coverage on bycatch estimates of the European purse seine fleet},
  colorlinks=true,
  linkcolor={blue},
  filecolor={Maroon},
  citecolor={Blue},
  urlcolor={Blue},
  pdfcreator={LaTeX via pandoc}}


\title{Impacts of the sampling coverage on bycatch estimates of the
European purse seine fleet}


\author[1]{Giancarlo M. Correa}
\author[1]{Jon Ruiz}
\author[2]{Miguel Herrera}
\author[3]{Nekane Alzorriz}

\affil[1]{AZTI, Marine Research, Basque Research and Technology Alliance
(BRTA), Txatxarramendi ugartea z/g, 48395 Sukarrieta (Bizkaia), Spain}
\affil[2]{Organización Productores Asociados Grandes Atuneros
Congeladores (OPAGAC), Cl. de Ayala, 54, 28001 Salamanca (Madrid),
Spain}
\affil[3]{Asociación Nacional de Armadores de Buques Atuneros
Congeladores (ANABAC), Txibitxiaga Kalea, 24, 48370 Bermeo (Bizkaia),
Spain}


\date{}
\begin{document}
\maketitle


\textbf{Abstract}

\emph{Keywords}

bycatch, pelagic species, spatiotemporal models, purse seine fleet,
tunas

\newpage

\section{Introduction}\label{introduction}

Purse seine fishing is a technique that targets and catch entire fish
schools in the surface by encircling them with a fishing net called
``seine''. In the tropical oceans, purse seine is employed to target
tropical tunas such as skipjack (\emph{Katsuwonus pelamis}), yellowfin
(\emph{Thunnus albacares}), and bigeye (\emph{Thunnus obesus}). In
addition to target species, tropical tuna fisheries catch non-target
species collectively referred to as bycatch, which can be discarded at
sea, retained to be sold on local markets, or consumed on board
(\citeproc{ref-hallMitigatingBycatchTuna2017}{Hall et al., 2017}). The
ratio of bycatch to target tuna catch in the purse seine fleet is
considered relatively low in comparison to other fishing gears, such as
longlines, that can result in substantial levels of bycatch
(\citeproc{ref-amandeBycatchEuropeanPurse2010}{Amandè et al., 2010};
\citeproc{ref-liuPreliminaryEstimatesBlue2008}{Liu et al., 2008};
\citeproc{ref-peatmanEstimatingTrendsMagnitudes2023}{Peatman et al.,
2023}). However, the impact on pelagic populations and ecosystems may be
important, especially for vulnerable long-lived species with low
reproductive rates
(\citeproc{ref-dulvyExtinctionRiskConservation2014}{Dulvy et al.,
2014}). Therefore, it is essential to conduct studies on bycatch rates
and their variability over space and time; however, they are often
complicated by a lack of bycatch data recorded in fisheries logbooks,
taxonomic identification of bycatch species, among other factors.

One of the most reliable sources of information to quantify the amount
of bycatch is the use of on-board trained scientific data collectors
(a.k.a. observers). When designing an observer sampling program, the
level of coverage required will depend on the objectives of the observer
program, which might vary from estimating bycatch of protected species,
to improving bycatch and catch data for assessment of fish populations,
to collecting biological data. In some cases, it may be necessary to
have an exact count of the total incidental mortality of bycatch
species, especially threatened or endangered species, so a 100\%
observer coverage may be needed. However, in most cases, a level of
100\% observer coverage is not attainable, then the coverage level
chosen must ensure that the total bycatch estimate is sufficiently
accurate and precise. Then, assuming these observed units are
representative of unobserved activity, design-based (e.g., ratio
estimators, Cochran
(\citeproc{ref-cochranSamplingTechniques1977}{1977})) or model-based
(e.g., generalized linear models, Coelho et al.
(\citeproc{ref-coelhoComparingGLMGLMM2020}{2020})) approaches can be
used to expand the observed bycatch to the remainder of the fishery. One
of the main features of ratio estimators is that they do not incorporate
a formal underlying statistical model (i.e.~are free of any assumptions
regarding data structure), and therefore are broadly used in fisheries
worldwide, including tuna purse seine fisheries
(\citeproc{ref-amandePrecisionBycatchEstimates2012}{Amandè et al.,
2012}; \citeproc{ref-amandeBycatchEuropeanPurse2010}{Amandè et al.,
2010}).

Despite their widespread use, there are a number of potential issues in
applying ratio estimators to estimate bycatch. First, using observed
catches of target species or any other measure of effort implicitly
makes an assumption about a linear relationship between non-target and
target catches
(\citeproc{ref-amandePrecisionBycatchEstimates2012}{Amandè et al.,
2012}; \citeproc{ref-fonteneauRelationshipCatchEffort2003}{Fonteneau and
Richard, 2003}). This may be unrealistic since the distribution of
catches of non-target species is often zero-inflated or has a small
number of observations containing extremely high values
(\citeproc{ref-ortizAlternativeErrorDistribution2004}{Ortiz and Arocha,
2004}), and the liner relationship may not hold
(\citeproc{ref-stockUtilitySpatialModelbased2019}{Stock et al., 2019}).
Second, the boundaries of strata used in a ratio estimator can be
somewhat arbitrary whenever poststratified boundaries are used. For
instance, Amandè et al.
(\citeproc{ref-amandeBycatchEuropeanPurse2010}{2010}) defined strata in
the Atlantic Ocean based on ecological features for estimating bycatch
of the tuna EU purse seine fishery, which may not be adequate for all
bycatch species. Third, for rare-event bycatch species, it is common for
zero bycatch events to be observed in a given year (ratio estimator is
equal to 0), and when bycatch events are observed, the ratio estimator
often delivers implausibly high estimates. Lastly, a final and related
point is that within each stratum, bycatch rates are assumed to be
uniform, while in reality they may vary by season, depth, or other
factors.

Spatiotemporal models are increasingly adopted in multiple fisheries
applications
(\citeproc{ref-ducharme-barthImpactsFisheriesdependentSpatial2022}{Ducharme-Barth
et al., 2022}; \citeproc{ref-grussSupportingStockAssessment2023}{Grüss
et al., 2023}; \citeproc{ref-thorsonGuidanceDecisionsUsing2019}{Thorson,
2019}), including undertaking bycatch analyses (e.g., Yan et al.
(\citeproc{ref-yanSpatiotemporalModelingBycatch2022}{2022})). These
models can provide detailed predictions for any location based on
spatial autocorrelation in the observations. However, they are also
complicated and require more data to generate robust predictions, which
make it unsuitable for data-poor fisheries. In the majority of cases,
spatially explicit model-based estimators have increased precision
relative to simpler estimators that assign observations to strata
(\citeproc{ref-thorsonGeostatisticalDeltageneralizedLinear2015}{Thorson
et al., 2015}; \citeproc{ref-thorsonAccountingSpaceTime2013}{Thorson and
Ward, 2013}). There are a number of additional advantages of spatial
models, including the ability to better quantify shifts in distribution
(\citeproc{ref-thorsonModelbasedInferenceEstimating2016}{Thorson et al.,
2016}) and improved ability to identify fine-scale hotspots of high
bycatch
(\citeproc{ref-cosandey-godinApplyingBayesianSpatiotemporal2015}{Cosandey-Godin
et al., 2015}). For tuna fisheries, spatiotemporal models like
generalized additive models (GAMs) have recently been used to obtain
annual estimates of the most important bycatch species
(\citeproc{ref-dumontModelingBycatchAbundance2024}{Dumont et al., 2024};
\citeproc{ref-peatmanEstimatingTrendsMagnitudes2023}{Peatman et al.,
2023}).

In this study, we implemented a simulation experiment to evaluate the
impacts of different levels of sampling coverage on the annual bycatch
estimates derived from design-based and model-based estimators. Our
hypothesis is that model-based estimators may provide more accurate
bycatch estimates under low sampling coverage scenarios. We used data
from the EU purse seine tuna fishery operating in the Atlantic Ocean as
study case. We performed our analyses differentiating by set type:
operating on free schools or floating objects, since they have different
bycatch dynamics
(\citeproc{ref-peatmanEstimatingTrendsMagnitudes2023}{Peatman et al.,
2023}). Our simulation experiment may be extended to other fisheries
with fine-scale bycatch information and support the implementation of
sampling programmes across tuna RFMOs.

\section{Methods}\label{methods}

We use geostatistical generalized linear mixed models (GLMMs), which
specify a geostatistical model to estimate a smoothed surface
representing spatial variation in the studied variable (e.g., observed
bycatch per unit of effort).

\section{References}\label{references}

\phantomsection\label{refs}
\begin{CSLReferences}{1}{0}
\bibitem[\citeproctext]{ref-amandeBycatchEuropeanPurse2010}
Amandè, M.J., Ariz, J., Chassot, E., De Molina, A.D., Gaertner, D.,
Murua, H., Pianet, R., Ruiz, J., Chavance, P., 2010. Bycatch of the
{European} purse seine tuna fishery in the {Atlantic Ocean} for the
2003--2007 period. Aquatic Living Resources 23, 353--362.
\url{https://doi.org/10.1051/alr/2011003}

\bibitem[\citeproctext]{ref-amandePrecisionBycatchEstimates2012}
Amandè, M.J., Chassot, E., Chavance, P., Murua, H., De Molina, A.D.,
Bez, N., 2012. Precision in bycatch estimates: The case of tuna
purse-seine fisheries in the {Indian Ocean}. ICES Journal of Marine
Science 69, 1501--1510. \url{https://doi.org/10.1093/icesjms/fss106}

\bibitem[\citeproctext]{ref-cochranSamplingTechniques1977}
Cochran, W.G., 1977. Sampling {Techniques}, 3rd ed. John Wiley \& Sons,
New York.

\bibitem[\citeproctext]{ref-coelhoComparingGLMGLMM2020}
Coelho, R., Infante, P., Santos, M.N., 2020. Comparing {GLM}, {GLMM},
and {GEE} modeling approaches for catch rates of bycatch species: {A}
case study of blue shark fisheries in the {South Atlantic}. Fisheries
Oceanography 29, 169--184. \url{https://doi.org/10.1111/fog.12462}

\bibitem[\citeproctext]{ref-cosandey-godinApplyingBayesianSpatiotemporal2015}
Cosandey-Godin, A., Krainski, E.T., Worm, B., Flemming, J.M., 2015.
Applying {Bayesian} spatiotemporal models to fisheries bycatch in the
{Canadian Arctic}. Canadian Journal of Fisheries and Aquatic Sciences
72, 186--197. \url{https://doi.org/10.1139/cjfas-2014-0159}

\bibitem[\citeproctext]{ref-ducharme-barthImpactsFisheriesdependentSpatial2022}
Ducharme-Barth, N.D., Grüss, A., Vincent, M.T., Kiyofuji, H., Aoki, Y.,
Pilling, G., Hampton, J., Thorson, J.T., 2022. Impacts of
fisheries-dependent spatial sampling patterns on catch-per-unit-effort
standardization: {A} simulation study and fishery application. Fisheries
Research 246, 106169.
\url{https://doi.org/10.1016/j.fishres.2021.106169}

\bibitem[\citeproctext]{ref-dulvyExtinctionRiskConservation2014}
Dulvy, N.K., Fowler, S.L., Musick, J.A., Cavanagh, R.D., Kyne, P.M.,
Harrison, L.R., Carlson, J.K., Davidson, L.N., Fordham, S.V., Francis,
M.P., Pollock, C.M., Simpfendorfer, C.A., Burgess, G.H., Carpenter,
K.E., Compagno, L.J., Ebert, D.A., Gibson, C., Heupel, M.R.,
Livingstone, S.R., Sanciangco, J.C., Stevens, J.D., Valenti, S., White,
W.T., 2014. Extinction risk and conservation of the world's sharks and
rays. eLife 3, e00590. \url{https://doi.org/10.7554/eLife.00590}

\bibitem[\citeproctext]{ref-dumontModelingBycatchAbundance2024}
Dumont, A., Duparc, A., Sabarros, P.S., Kaplan, D.M., 2024. Modeling
bycatch abundance in tropical tuna purse seine fisheries on floating
objects using the {\(\Delta\)} method. ICES Journal of Marine Science
81, 887--908. \url{https://doi.org/10.1093/icesjms/fsae043}

\bibitem[\citeproctext]{ref-fonteneauRelationshipCatchEffort2003}
Fonteneau, A., Richard, N., 2003. Relationship between catch, effort,
{CPUE} and local abundance for non-target species, such as billfishes,
caught by {Indian Ocean} longline fisheries. Marine and Freshwater
Research 54, 383--392. \url{https://doi.org/10.1071/MF01268}

\bibitem[\citeproctext]{ref-grussSupportingStockAssessment2023}
Grüss, A., McKenzie, J.R., Lindegren, M., Bian, R., Hoyle, S.D., Devine,
J.A., 2023. Supporting a stock assessment with spatio-temporal models
fitted to fisheries-dependent data. Fisheries Research 262, 106649.
\url{https://doi.org/10.1016/j.fishres.2023.106649}

\bibitem[\citeproctext]{ref-hallMitigatingBycatchTuna2017}
Hall, M., Gilman, E., Minami, H., Mituhasi, T., Carruthers, E., 2017.
Mitigating bycatch in tuna fisheries. Reviews in Fish Biology and
Fisheries 27, 881--908. \url{https://doi.org/10.1007/s11160-017-9478-x}

\bibitem[\citeproctext]{ref-liuPreliminaryEstimatesBlue2008}
Liu, K.-M., Tsai, W.-P., Joung, S.-J., 2008. Preliminary estimates of
blue and mako sharks bycatch and cpue of {Taiwanese} longline fishery in
the {Atlantic Ocean} (No. SCRS/2008/153). ICCAT (International
Commission for the Conservation of Atlantic Tunas), Madrid, Spain.

\bibitem[\citeproctext]{ref-ortizAlternativeErrorDistribution2004}
Ortiz, M., Arocha, F., 2004. Alternative error distribution models for
standardization of catch rates of non-target species from a pelagic
longline fishery: Billfish species in the {Venezuelan} tuna longline
fishery. Fisheries Research 70, 275--297.
\url{https://doi.org/10.1016/j.fishres.2004.08.028}

\bibitem[\citeproctext]{ref-peatmanEstimatingTrendsMagnitudes2023}
Peatman, T., Allain, V., Bell, L., Muller, B., Panizza, A., Phillip,
N.B., Pilling, G., Nicol, S., 2023. Estimating trends and magnitudes of
bycatch in the tuna fisheries of the {Western} and {Central Pacific
Ocean}. Fish and Fisheries 24, 812--828.
\url{https://doi.org/10.1111/faf.12771}

\bibitem[\citeproctext]{ref-stockUtilitySpatialModelbased2019}
Stock, B.C., Ward, E.J., Thorson, J.T., Jannot, J.E., Semmens, B.X.,
2019. The utility of spatial model-based estimators of unobserved
bycatch. ICES Journal of Marine Science 76, 255--267.
\url{https://doi.org/10.1093/icesjms/fsy153}

\bibitem[\citeproctext]{ref-thorsonGuidanceDecisionsUsing2019}
Thorson, J.T., 2019. Guidance for decisions using the {Vector
Autoregressive Spatio-Temporal} ({VAST}) package in stock, ecosystem,
habitat and climate assessments. Fisheries Research 210, 143--161.
\url{https://doi.org/10.1016/j.fishres.2018.10.013}

\bibitem[\citeproctext]{ref-thorsonModelbasedInferenceEstimating2016}
Thorson, J.T., Pinsky, M.L., Ward, E.J., 2016. Model-based inference for
estimating shifts in species distribution, area occupied and centre of
gravity. Methods in Ecology and Evolution 7, 990--1002.
\url{https://doi.org/10.1111/2041-210X.12567}

\bibitem[\citeproctext]{ref-thorsonGeostatisticalDeltageneralizedLinear2015}
Thorson, J.T., Shelton, A.O., Ward, E.J., Skaug, H.J., 2015.
Geostatistical delta-generalized linear mixed models improve precision
for estimated abundance indices for {West Coast} groundfishes. ICES
Journal of Marine Science 72, 1297--1310.
\url{https://doi.org/10.1093/icesjms/fsu243}

\bibitem[\citeproctext]{ref-thorsonAccountingSpaceTime2013}
Thorson, J.T., Ward, E.J., 2013. Accounting for space--time interactions
in index standardization models. Fisheries Research 147, 426--433.
\url{https://doi.org/10.1016/j.fishres.2013.03.012}

\bibitem[\citeproctext]{ref-yanSpatiotemporalModelingBycatch2022}
Yan, Y., Cantoni, E., Field, C., Treble, M., Flemming, J.M., 2022.
Spatiotemporal modeling of bycatch data: Methods and a practical guide
through a case study in a {Canadian Arctic} fishery. Canadian Journal of
Fisheries and Aquatic Sciences 79, 148--158.
\url{https://doi.org/10.1139/cjfas-2020-0267}

\end{CSLReferences}

\newpage{}

\section{Tables}\label{tables}




\end{document}
