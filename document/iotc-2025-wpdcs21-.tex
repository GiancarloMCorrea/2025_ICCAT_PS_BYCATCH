% Options for packages loaded elsewhere
% Options for packages loaded elsewhere
\PassOptionsToPackage{unicode}{hyperref}
\PassOptionsToPackage{hyphens}{url}
\PassOptionsToPackage{dvipsnames,svgnames,x11names}{xcolor}
%
\documentclass[
]{scrartcl}
\usepackage{xcolor}
\usepackage[left=2.5cm,right=2.5cm,top=3cm,bottom=3cm]{geometry}
\usepackage{amsmath,amssymb}
\setcounter{secnumdepth}{5}
\usepackage{iftex}
\ifPDFTeX
  \usepackage[T1]{fontenc}
  \usepackage[utf8]{inputenc}
  \usepackage{textcomp} % provide euro and other symbols
\else % if luatex or xetex
  \usepackage{unicode-math} % this also loads fontspec
  \defaultfontfeatures{Scale=MatchLowercase}
  \defaultfontfeatures[\rmfamily]{Ligatures=TeX,Scale=1}
\fi
\usepackage{lmodern}
\ifPDFTeX\else
  % xetex/luatex font selection
\fi
% Use upquote if available, for straight quotes in verbatim environments
\IfFileExists{upquote.sty}{\usepackage{upquote}}{}
\IfFileExists{microtype.sty}{% use microtype if available
  \usepackage[]{microtype}
  \UseMicrotypeSet[protrusion]{basicmath} % disable protrusion for tt fonts
}{}
\makeatletter
\@ifundefined{KOMAClassName}{% if non-KOMA class
  \IfFileExists{parskip.sty}{%
    \usepackage{parskip}
  }{% else
    \setlength{\parindent}{0pt}
    \setlength{\parskip}{6pt plus 2pt minus 1pt}}
}{% if KOMA class
  \KOMAoptions{parskip=half}}
\makeatother
% Make \paragraph and \subparagraph free-standing
\makeatletter
\ifx\paragraph\undefined\else
  \let\oldparagraph\paragraph
  \renewcommand{\paragraph}{
    \@ifstar
      \xxxParagraphStar
      \xxxParagraphNoStar
  }
  \newcommand{\xxxParagraphStar}[1]{\oldparagraph*{#1}\mbox{}}
  \newcommand{\xxxParagraphNoStar}[1]{\oldparagraph{#1}\mbox{}}
\fi
\ifx\subparagraph\undefined\else
  \let\oldsubparagraph\subparagraph
  \renewcommand{\subparagraph}{
    \@ifstar
      \xxxSubParagraphStar
      \xxxSubParagraphNoStar
  }
  \newcommand{\xxxSubParagraphStar}[1]{\oldsubparagraph*{#1}\mbox{}}
  \newcommand{\xxxSubParagraphNoStar}[1]{\oldsubparagraph{#1}\mbox{}}
\fi
\makeatother


\usepackage{longtable,booktabs,array}
\usepackage{calc} % for calculating minipage widths
% Correct order of tables after \paragraph or \subparagraph
\usepackage{etoolbox}
\makeatletter
\patchcmd\longtable{\par}{\if@noskipsec\mbox{}\fi\par}{}{}
\makeatother
% Allow footnotes in longtable head/foot
\IfFileExists{footnotehyper.sty}{\usepackage{footnotehyper}}{\usepackage{footnote}}
\makesavenoteenv{longtable}
\usepackage{graphicx}
\makeatletter
\newsavebox\pandoc@box
\newcommand*\pandocbounded[1]{% scales image to fit in text height/width
  \sbox\pandoc@box{#1}%
  \Gscale@div\@tempa{\textheight}{\dimexpr\ht\pandoc@box+\dp\pandoc@box\relax}%
  \Gscale@div\@tempb{\linewidth}{\wd\pandoc@box}%
  \ifdim\@tempb\p@<\@tempa\p@\let\@tempa\@tempb\fi% select the smaller of both
  \ifdim\@tempa\p@<\p@\scalebox{\@tempa}{\usebox\pandoc@box}%
  \else\usebox{\pandoc@box}%
  \fi%
}
% Set default figure placement to htbp
\def\fps@figure{htbp}
\makeatother


% definitions for citeproc citations
\NewDocumentCommand\citeproctext{}{}
\NewDocumentCommand\citeproc{mm}{%
  \begingroup\def\citeproctext{#2}\cite{#1}\endgroup}
\makeatletter
 % allow citations to break across lines
 \let\@cite@ofmt\@firstofone
 % avoid brackets around text for \cite:
 \def\@biblabel#1{}
 \def\@cite#1#2{{#1\if@tempswa , #2\fi}}
\makeatother
\newlength{\cslhangindent}
\setlength{\cslhangindent}{1.5em}
\newlength{\csllabelwidth}
\setlength{\csllabelwidth}{3em}
\newenvironment{CSLReferences}[2] % #1 hanging-indent, #2 entry-spacing
 {\begin{list}{}{%
  \setlength{\itemindent}{0pt}
  \setlength{\leftmargin}{0pt}
  \setlength{\parsep}{0pt}
  % turn on hanging indent if param 1 is 1
  \ifodd #1
   \setlength{\leftmargin}{\cslhangindent}
   \setlength{\itemindent}{-1\cslhangindent}
  \fi
  % set entry spacing
  \setlength{\itemsep}{#2\baselineskip}}}
 {\end{list}}
\usepackage{calc}
\newcommand{\CSLBlock}[1]{\hfill\break\parbox[t]{\linewidth}{\strut\ignorespaces#1\strut}}
\newcommand{\CSLLeftMargin}[1]{\parbox[t]{\csllabelwidth}{\strut#1\strut}}
\newcommand{\CSLRightInline}[1]{\parbox[t]{\linewidth - \csllabelwidth}{\strut#1\strut}}
\newcommand{\CSLIndent}[1]{\hspace{\cslhangindent}#1}



\setlength{\emergencystretch}{3em} % prevent overfull lines

\providecommand{\tightlist}{%
  \setlength{\itemsep}{0pt}\setlength{\parskip}{0pt}}



 


\usepackage[noblocks]{authblk}
\renewcommand*{\Authsep}{, }
\renewcommand*{\Authand}{, }
\renewcommand*{\Authands}{, }
\renewcommand\Affilfont{\small}
\usepackage{lineno}
\usepackage{scrlayer-scrpage}
\rohead{IOTC-2025-WPDCS21-}
\usepackage{lscape}
\newcommand{\blandscape}{\begin{landscape}}
\newcommand{\elandscape}{\end{landscape}}
\makeatletter
\@ifpackageloaded{caption}{}{\usepackage{caption}}
\AtBeginDocument{%
\ifdefined\contentsname
  \renewcommand*\contentsname{Table of contents}
\else
  \newcommand\contentsname{Table of contents}
\fi
\ifdefined\listfigurename
  \renewcommand*\listfigurename{List of Figures}
\else
  \newcommand\listfigurename{List of Figures}
\fi
\ifdefined\listtablename
  \renewcommand*\listtablename{List of Tables}
\else
  \newcommand\listtablename{List of Tables}
\fi
\ifdefined\figurename
  \renewcommand*\figurename{Figure}
\else
  \newcommand\figurename{Figure}
\fi
\ifdefined\tablename
  \renewcommand*\tablename{Table}
\else
  \newcommand\tablename{Table}
\fi
}
\@ifpackageloaded{float}{}{\usepackage{float}}
\floatstyle{ruled}
\@ifundefined{c@chapter}{\newfloat{codelisting}{h}{lop}}{\newfloat{codelisting}{h}{lop}[chapter]}
\floatname{codelisting}{Listing}
\newcommand*\listoflistings{\listof{codelisting}{List of Listings}}
\makeatother
\makeatletter
\makeatother
\makeatletter
\@ifpackageloaded{caption}{}{\usepackage{caption}}
\@ifpackageloaded{subcaption}{}{\usepackage{subcaption}}
\makeatother
\usepackage{bookmark}
\IfFileExists{xurl.sty}{\usepackage{xurl}}{} % add URL line breaks if available
\urlstyle{same}
\hypersetup{
  pdftitle={How Observation Coverage Shapes Bycatch Metrics in the Tropical Tuna Purse Seine Fishery},
  colorlinks=true,
  linkcolor={blue},
  filecolor={Maroon},
  citecolor={Blue},
  urlcolor={Blue},
  pdfcreator={LaTeX via pandoc}}


\title{How Observation Coverage Shapes Bycatch Metrics in the Tropical
Tuna Purse Seine Fishery}


\author[1]{Giancarlo M. Correa}
\author[1]{Jon Ruiz}
\author[2]{María Lourdes Ramos}
\author[1]{Arnaitz Mugerza}
\author[3]{Miguel Herrera}
\author[4]{Nekane Alzorriz}

\affil[1]{AZTI, Marine Research, Basque Research and Technology Alliance
(BRTA), Txatxarramendi ugartea z/g, 48395 Sukarrieta, Bizkaia, Spain}
\affil[2]{Centro Oceanográfico de Canarias, Instituto Español de
Oceanografía (IEO-CSIC), C. Farola del Mar 22, 38180 San Andrés, Santa
Cruz de Tenerife, Spain}
\affil[3]{Organización Productores Asociados Grandes Atuneros
Congeladores (OPAGAC-AGAC), Cl. de Ayala, 54, 28001 Salamanca, Madrid,
Spain}
\affil[4]{Asociación Nacional de Armadores de Buques Atuneros
Congeladores (ANABAC), Txibitxiaga Kalea, 24, 48370 Bermeo, Bizkaia,
Spain}


\date{}
\begin{document}
\maketitle


\textbf{Abstract}

\emph{Keywords}

bycatch, pelagic species, spatiotemporal models, purse seine fleet,
tunas

\newpage

\section{Introduction}\label{introduction}

Purse seine fishing is a technique that targets and catch entire fish
schools in the surface by encircling them with a fishing net called
``seine''. In the tropical oceans, purse seine is employed to target
tropical tunas such as skipjack (\emph{Katsuwonus pelamis}), yellowfin
(\emph{Thunnus albacares}), and bigeye (\emph{Thunnus obesus}). In
addition to target species, tropical tuna fisheries catch non-target
species collectively referred to as bycatch, which can be discarded at
sea, retained to be sold on local markets, or consumed on board
(\citeproc{ref-hallMitigatingBycatchTuna2017}{Hall et al., 2017}). The
ratio of bycatch to target tuna catch in the purse seine fleet is
considered relatively low in comparison to other fishing gears, such as
longlines, that can result in substantial levels of bycatch
(\citeproc{ref-amandeBycatchEuropeanPurse2010}{Amandè et al., 2010};
\citeproc{ref-liuPreliminaryEstimatesBlue2008}{Liu et al., 2008};
\citeproc{ref-peatmanEstimatingTrendsMagnitudes2023}{Peatman et al.,
2023}). However, the impact on pelagic populations and ecosystems may be
important, especially for vulnerable long-lived species with low
reproductive rates
(\citeproc{ref-dulvyExtinctionRiskConservation2014}{Dulvy et al.,
2014}). Therefore, it is essential to conduct studies on bycatch rates
and their variability over space and time; however, they are often
complicated by a lack of bycatch data recorded in fisheries logbooks,
taxonomic identification of bycatch species, among other factors.

One of the most reliable sources of information to quantify the amount
of bycatch is the use of on-board trained scientific data collectors
(a.k.a. observers). When designing an observer sampling program, the
level of coverage required will depend on the objectives of the observer
program, which might vary from estimating bycatch of protected species,
to improving bycatch and catch data for assessment of fish populations,
to collecting biological data. In some cases, it may be necessary to
have an exact count of the total incidental mortality of bycatch
species, especially threatened or endangered species, so a 100\%
observer coverage may be needed. However, in most cases, a level of
100\% observer coverage is not attainable, then the coverage level
chosen must ensure that the total bycatch estimate is sufficiently
accurate and precise. Then, assuming these observed units are
representative of unobserved activity, design-based (e.g., ratio
estimators, Cochran
(\citeproc{ref-cochranSamplingTechniques1977}{1977})) or model-based
(e.g., generalized linear models, Coelho et al.
(\citeproc{ref-coelhoComparingGLMGLMM2020}{2020})) approaches can be
used to expand the observed bycatch to the remainder of the fishery. One
of the main features of ratio estimators is that they do not incorporate
a formal underlying statistical model (i.e.~are free of any assumptions
regarding data structure), and therefore are broadly used in fisheries
worldwide, including tuna purse seine fisheries
(\citeproc{ref-amandePrecisionBycatchEstimates2012}{Amandè et al.,
2012}; \citeproc{ref-amandeBycatchEuropeanPurse2010}{Amandè et al.,
2010}).

Despite their widespread use, there are a number of potential issues in
applying ratio estimators to estimate bycatch. First, using observed
catches of target species or any other measure of effort implicitly
makes an assumption about a linear relationship between non-target and
target catches
(\citeproc{ref-amandePrecisionBycatchEstimates2012}{Amandè et al.,
2012}; \citeproc{ref-fonteneauRelationshipCatchEffort2003}{Fonteneau and
Richard, 2003}). This may be unrealistic since the distribution of
catches of non-target species is often zero-inflated or has a small
number of observations containing extremely high values
(\citeproc{ref-ortizAlternativeErrorDistribution2004}{Ortiz and Arocha,
2004}), and the liner relationship may not hold
(\citeproc{ref-stockUtilitySpatialModelbased2019}{Stock et al., 2019}).
Second, the boundaries of strata used in a ratio estimator can be
somewhat arbitrary whenever poststratified boundaries are used. For
instance, Amandè et al.
(\citeproc{ref-amandeBycatchEuropeanPurse2010}{2010}) defined strata in
the Atlantic Ocean based on ecological features for estimating bycatch
of the tuna EU purse seine fishery, which may not be adequate for all
bycatch species. Third, for rare-event bycatch species, it is common for
zero bycatch events to be observed in a given year (ratio estimator is
equal to 0), and when bycatch events are observed, the ratio estimator
often delivers implausibly high estimates. Lastly, a final and related
point is that within each stratum, bycatch rates are assumed to be
uniform, while in reality they may vary by season, depth, or other
factors.

Spatiotemporal models are increasingly adopted in multiple fisheries
applications
(\citeproc{ref-ducharme-barthImpactsFisheriesdependentSpatial2022}{Ducharme-Barth
et al., 2022}; \citeproc{ref-grussSupportingStockAssessment2023}{Grüss
et al., 2023}; \citeproc{ref-thorsonGuidanceDecisionsUsing2019}{Thorson,
2019}), including undertaking bycatch analyses (e.g., Yan et al.
(\citeproc{ref-yanSpatiotemporalModelingBycatch2022}{2022})). These
models can provide detailed predictions for any location based on
spatial autocorrelation in the observations. However, they are also
complicated and require more data to generate robust predictions, which
make it unsuitable for data-poor fisheries. In the majority of cases,
spatially explicit model-based estimators have increased precision
relative to simpler estimators that assign observations to strata
(\citeproc{ref-thorsonGeostatisticalDeltageneralizedLinear2015}{Thorson
et al., 2015}; \citeproc{ref-thorsonAccountingSpaceTime2013}{Thorson and
Ward, 2013}). There are a number of additional advantages of spatial
models, including the ability to better quantify shifts in distribution
(\citeproc{ref-thorsonModelbasedInferenceEstimating2016}{Thorson et al.,
2016}) and improved ability to identify fine-scale hotspots of high
bycatch
(\citeproc{ref-cosandey-godinApplyingBayesianSpatiotemporal2015}{Cosandey-Godin
et al., 2015}). For tuna fisheries, spatiotemporal models like
generalized additive models (GAMs) have recently been used to obtain
annual estimates of the most important bycatch species
(\citeproc{ref-dumontModelingBycatchAbundance2024}{Dumont et al., 2024};
\citeproc{ref-peatmanEstimatingTrendsMagnitudes2023}{Peatman et al.,
2023}).

In this study, we implemented a simulation experiment to evaluate the
impacts of different levels of sampling coverage on the annual bycatch
estimates derived from design-based and model-based estimators. Our
hypothesis is that model-based estimators may provide more accurate
bycatch estimates under low sampling coverage scenarios. We used data
from the Spanish purse seine tuna fishery operating in the Atlantic
Ocean as study case. We performed our analyses differentiating by set
type: operating on free schools (FSC) or floating objects (FOB), since
they have different bycatch dynamics
(\citeproc{ref-peatmanEstimatingTrendsMagnitudes2023}{Peatman et al.,
2023}). Our simulation experiment may be extended to other fisheries
with fine-scale bycatch information and support the implementation of
sampling programmes across tuna RFMOs.

\section{Methods}\label{methods}

In the following sections, we describe the data used in our analyses,
the design and model-based estimators, and the simulation framework. The
analyses described below were performed by set type independently.

\subsection{Data}\label{data}

Our analyses use data collected by scientific observers aboard tropical
purse seine vessels operating in the Atlantic Ocean between 2015 and
2023 (\emph{observers data}). The dataset includes records from both the
Spanish scientific monitoring program (EU Data Collection Framework) and
the industry-funded Best Practices program, covering Spanish vessels and
those under other flags affiliated with ANABAC and OPAGAC.

Regardless of the monitoring program, observers followed a standardized
protocol. They recorded detailed information for each fishing set,
including estimates of target tuna catch and bycatch. For larger bycatch
species such as elasmobranchs and billfish, all individuals were
counted. For smaller, more abundant species, estimates were often based
on visual assessments. In addition to counts, observers conducted length
sampling to convert numbers into biomass using species-specific
length-weight relationships. Taxa were identified to the species level
whenever possible, although in rare cases, only higher taxonomic groups
(e.g., family) were recorded.

The \emph{effort data} used in our analyses pertain exclusively to the
Spanish fleet for the same period and fishing ground. This information
is sourced from the logbooks completed by the captains, who are required
to record details of each fishing trip, including the location and date
of every fishing operation.

This study does not present results for all taxa found in the purse
seine fleet's bycatch. Instead, a targeted selection of species or
groups of species was made to represent three categories: the most
abundant taxa, the rare or less prevalent ones, and those considered
vulnerable or of special interest. The list of these taxa are presented
in Table XX.

\subsection{Model fitting}\label{sec-fit}

We use the \emph{sdmTMB} geostatistical spatiotemporal model
(\citeproc{ref-Anderson2024}{Anderson et al., 2024}) to fit
taxon-specific bycatch per set (in weight) using the observers data.
Geostatistical spatiotemporal models have become widely used in
fisheries over the last decade (\citeproc{ref-Anderson2024}{Anderson et
al., 2024}; \citeproc{ref-thorsonGuidanceDecisionsUsing2019}{Thorson,
2019}), and are used when georeferenced data (e.g.~each has a
corresponding latitude and longitude) with an underlying spatial process
is available. \emph{sdmTMB} is written in Template Model Builder (TMB,
Kristensen et al.
(\citeproc{ref-kristensenTMBAutomaticDifferentiation2016}{2016})) and R
(\citeproc{ref-rcoreteamLanguageEnvironmentStatistical2025}{Team, 2025})
for a friendly user interface, and can be viewed as an extension of
generalized linear mixed models (GLMMs), but with additional spatial and
spatiotemporal components, which are approximated as random effects.

Mathematically, the model structure can be expressed as:

\[u_{s=1:S,t}=f^{-1}(Xb+\omega+\epsilon_t)\]

where \(\u\) represents the bycatch predictions at locations \(s=1:S\)
for a given year \(t\), \(f^{-1}()\) is the inverse link function, \(X\)
represents the design matrix of fixed effects, \(b\) is the vector of
estimated parameters, \(\omega\) are the estimated latent spatial
effects, and \(\epsilon_t\) represents year-to-year latent
spatiotemporal effects. \(\omega\) represents a spatial intercept that
is constant with time while \(\epsilon_t\) represents spatial deviations
over time, and both are modelled as Gaussian random fields (GRFs):

\[\omega \sim MVN(0,\Sigma_\omega)\]
\[\epsilon \sim MVN(0,\Sigma_\epsilon)\]

where \(MVN\) is the multivariate normal distribution, and the
covariance matrix \(\Sigma\) is modelled with Matérn covariance
(\citeproc{ref-lindgrenExplicitLinkGaussian2011}{Lindgren et al., 2011};
\citeproc{ref-maternSpatialVariation1986}{Matérn, 1986}), which defines
the rate at which spatial covariance decays with distance. \emph{sdmTMB}
approximates the GRF by relying on the Stochastic Partial Differential
Equation (SPDE) approach using the Integrated Nested Laplace
Approximation in \emph{R-INLA} to reduce computational costs
(\citeproc{ref-rueApproximateBayesianInference2009}{Rue et al., 2009}).
The first step when using the SPDE approach is to construct the mesh,
which, in our case, was composed of triangles covering the studied area
with a minimum allowed triangle edge length (\emph{cutoff}) of 1.5
degrees.

For some taxa, especially the less recurrent ones, the inclusion of both
the spatial and spatiotemporal terms caused the model failed to
converge. For those cases, we reran the model only including the spatial
term in order to simplify the model structure.

We used the Tweedie distribution \(Tweedie(\mu, \phi^2,p)\), where
\(1<p<2\), and a log link function
(\citeproc{ref-tweedieIndexWhichDistinguishes1984}{Tweedie, 1984}). The
Tweedie model is an extension of compound Poisson model derived from the
stochastic process where the weight of the response variable (e.g.,
catch data) has a gamma distribution and has an advantage of handling
the zero-catch data in a unified way
(\citeproc{ref-shonoApplicationTweedieDistribution2008}{Shono, 2008}).
For fixed effects, we incorporated the year and quarter effects as
factors, and the target tuna catch (the sum of skipjack, yellowfin, and
bigeye tunas) as a continuous covariate.

For all fitted models, we checked that the maximum gradient was smaller
than 1e-03, the Hessian was invertible, and standard errors were
estimated for all fixed effects and did not look unreasonably large
(``safety checks''). We then used the \emph{DHARMa} R package
(\citeproc{ref-hartigDHARMaResidualDiagnostics2022}{Hartig, 2022}) to
evaluate the model residuals. Standard raw residuals are not always
appropriate when using generalized linear models, and other types of
residuals are commonly used instead. \emph{DHARMa} uses a
simulation-based approach to create readily interpretable scaled
(quantile) residuals for generalized linear mixed models. We analyzed
two plots produced by \emph{DHARMa}: 1) the QQ plot residuals, which
detects overall deviations from the expected distribution, and 2) the
residual vs.~predicted plot, which detects trends in residuals along
model predictions and simulation outliers.

\subsection{Simulation}\label{sec-sim}

One of the advantages of using models like \emph{sdmTMB} is that we can
simulate new observations using a new dataset (``prediction dataset'')
containing the same covariates used when fitting the model. In our case,
we used the effort data as the prediction dataset and simulated new
observations (i.e., bycatch in weight for every fishing set in the
effort data) using the fitted models for each taxa in
Section~\ref{sec-fit}. These simulated observations keep the statistical
properties of the original bycatch data. We refer to the effort data
with simulated bycatch observations as the \emph{simulated data}.

We then took a subset of the simulated data with different sampling
coverage scenarios: 5\%, 10\%, 20\%, 30\%, 40\%, 50\%, 70\%, and 90\%.
To approximate real-case situations, we performed this subsetting
stratified by year, and then selected the fishing trips observed under
that sampling coverage scenario. The obtained \emph{sampled data}
represent the observers data that would have been obtained from an
observers program with the specified sampling coverage.

Then, using the sampled data, we estimated the annual bycatch using two
approaches: ratio and model-based estimator, which are described below.

\subsubsection{Ratio estimator}\label{ratio-estimator}

We used the spatially-stratified bycatch--over-target catch ratio. For a
given taxon, the ratio (\(R_{y,a}\)) was calculated for every defined
\(5\times 5^\circ\) grid \(a\) in the study area and year \(y\) as
follows:

\[R_{y,a} = \frac{B_{y,a}}{T_{y,a}}\]

where \(BY_{y,a}\) is the total bycatch and \(TG_{y,a}\) the total
tropical tuna catch obtained from the sampled data. In a few cases,
there could happen that \(T_{y,a}=0\) (e.g., sets with target catch
equal to zero or ``null sets''), so \(R_{y,a}\) could not be calculated.
Therefore, exclusively for those cases, we assumed that \(R_{y,a}=0\).

Then, assuming a linear relationship between bycatch and target catch,
we calculated the total bycatch:

\[\hat{B}_{y,a} = R_{y,a}T^\ast_{y,a}\]

where \(T^\ast_{y,a}\) is the total target catch in grid \(a\) and year
\(y\) obtained from the effort data. Especially for low sampling
coverage scenarios, it is expected to have missing \(R_{y,a}\) values
for some grids due to the sampled data do not cover all the grids in the
study area. Therefore, exclusively for those grids, \(R_{y,a}\) was
calculated using \(B_{y}/T_{y}\), where \(B_{y}\) and \(T_{y}\) are the
total bycatch and target catch in the whole area, respectively, derived
from the information in the sampled data.

Finally, the annual bycatch estimate is calculated:
\(\hat{B}_{y}=\sum_a \hat{B}_{y,a}\).

\subsubsection{Model-based estimator}\label{model-based-estimator}

We followed the same modelling framework described in
Section~\ref{sec-fit}. Once the fitted model is obtained, we then made
predictions using the effort data, which generated predicted bycatch
observations for every fishing set in that dataset. Then, we summed the
predicted bycatch values per year to estimate the annual bycatch
\(\hat{B}_{y}\).

Especially when the sampling coverage is low, we could find cases when a
given taxon is not detected in the sampled data (i.e., bycatch equal to
zero for all fishing sets). In those cases, the model-based estimator
was not run and we assumed \(\hat{B}_{y}=0\) for all years. Another
special case is when the model did not pass the safety checks (see
Section~\ref{sec-fit}). In those cases, we were not able to produce
bycatch estimates \(\hat{B}_{y}\) and reported the rate of model
failure.

\subsection{Performance}\label{performance}

The procedure explained in Section~\ref{sec-sim} was repeated 100 times
(``replicates'') with different seeds to produce the simulated data,
therefore we obtained 100 annual bycatch estimates by the ratio and
model-based estimators for each taxa \(q\). We calculated the relative
error for every replicate \(i\):
\(RE_{i,q,y}=(\hat{B}_{i,q,y} - B_{i,q,y})/B_{i,q,y}\), where
\(B_{i,q,y}\) represents the true annual bycatch obtained from the
simulated data. The width of the 95\% quantile of RE over replicates was
used as a measure of precision and the median as a proxy of bias, and
these metrics were used to compare the performance of the ratio and
model-based estimators.

\section{References}\label{references}

\phantomsection\label{refs}
\begin{CSLReferences}{1}{0}
\bibitem[\citeproctext]{ref-amandeBycatchEuropeanPurse2010}
Amandè, M.J., Ariz, J., Chassot, E., De Molina, A.D., Gaertner, D.,
Murua, H., Pianet, R., Ruiz, J., Chavance, P., 2010. Bycatch of the
{European} purse seine tuna fishery in the {Atlantic Ocean} for the
2003--2007 period. Aquatic Living Resources 23, 353--362.
\url{https://doi.org/10.1051/alr/2011003}

\bibitem[\citeproctext]{ref-amandePrecisionBycatchEstimates2012}
Amandè, M.J., Chassot, E., Chavance, P., Murua, H., De Molina, A.D.,
Bez, N., 2012. Precision in bycatch estimates: The case of tuna
purse-seine fisheries in the {Indian Ocean}. ICES Journal of Marine
Science 69, 1501--1510. \url{https://doi.org/10.1093/icesjms/fss106}

\bibitem[\citeproctext]{ref-Anderson2024}
Anderson, S.C., Ward, E.J., English, P.A., Barnett, L.A.K., Thorson,
J.T., 2024. {sdmTMB}: {An R} package for fast, flexible, and
user-friendly generalized linear mixed effects models with spatial and
spatiotemporal random fields. bioRxiv : the preprint server for biology.
\url{https://doi.org/10.1101/2022.03.24.485545}

\bibitem[\citeproctext]{ref-cochranSamplingTechniques1977}
Cochran, W.G., 1977. Sampling {Techniques}, 3rd ed. John Wiley \& Sons,
New York.

\bibitem[\citeproctext]{ref-coelhoComparingGLMGLMM2020}
Coelho, R., Infante, P., Santos, M.N., 2020. Comparing {GLM}, {GLMM},
and {GEE} modeling approaches for catch rates of bycatch species: {A}
case study of blue shark fisheries in the {South Atlantic}. Fisheries
Oceanography 29, 169--184. \url{https://doi.org/10.1111/fog.12462}

\bibitem[\citeproctext]{ref-cosandey-godinApplyingBayesianSpatiotemporal2015}
Cosandey-Godin, A., Krainski, E.T., Worm, B., Flemming, J.M., 2015.
Applying {Bayesian} spatiotemporal models to fisheries bycatch in the
{Canadian Arctic}. Canadian Journal of Fisheries and Aquatic Sciences
72, 186--197. \url{https://doi.org/10.1139/cjfas-2014-0159}

\bibitem[\citeproctext]{ref-ducharme-barthImpactsFisheriesdependentSpatial2022}
Ducharme-Barth, N.D., Grüss, A., Vincent, M.T., Kiyofuji, H., Aoki, Y.,
Pilling, G., Hampton, J., Thorson, J.T., 2022. Impacts of
fisheries-dependent spatial sampling patterns on catch-per-unit-effort
standardization: {A} simulation study and fishery application. Fisheries
Research 246, 106169.
\url{https://doi.org/10.1016/j.fishres.2021.106169}

\bibitem[\citeproctext]{ref-dulvyExtinctionRiskConservation2014}
Dulvy, N.K., Fowler, S.L., Musick, J.A., Cavanagh, R.D., Kyne, P.M.,
Harrison, L.R., Carlson, J.K., Davidson, L.N., Fordham, S.V., Francis,
M.P., Pollock, C.M., Simpfendorfer, C.A., Burgess, G.H., Carpenter,
K.E., Compagno, L.J., Ebert, D.A., Gibson, C., Heupel, M.R.,
Livingstone, S.R., Sanciangco, J.C., Stevens, J.D., Valenti, S., White,
W.T., 2014. Extinction risk and conservation of the world's sharks and
rays. eLife 3, e00590. \url{https://doi.org/10.7554/eLife.00590}

\bibitem[\citeproctext]{ref-dumontModelingBycatchAbundance2024}
Dumont, A., Duparc, A., Sabarros, P.S., Kaplan, D.M., 2024. Modeling
bycatch abundance in tropical tuna purse seine fisheries on floating
objects using the {\(\Delta\)} method. ICES Journal of Marine Science
81, 887--908. \url{https://doi.org/10.1093/icesjms/fsae043}

\bibitem[\citeproctext]{ref-fonteneauRelationshipCatchEffort2003}
Fonteneau, A., Richard, N., 2003. Relationship between catch, effort,
{CPUE} and local abundance for non-target species, such as billfishes,
caught by {Indian Ocean} longline fisheries. Marine and Freshwater
Research 54, 383--392. \url{https://doi.org/10.1071/MF01268}

\bibitem[\citeproctext]{ref-grussSupportingStockAssessment2023}
Grüss, A., McKenzie, J.R., Lindegren, M., Bian, R., Hoyle, S.D., Devine,
J.A., 2023. Supporting a stock assessment with spatio-temporal models
fitted to fisheries-dependent data. Fisheries Research 262, 106649.
\url{https://doi.org/10.1016/j.fishres.2023.106649}

\bibitem[\citeproctext]{ref-hallMitigatingBycatchTuna2017}
Hall, M., Gilman, E., Minami, H., Mituhasi, T., Carruthers, E., 2017.
Mitigating bycatch in tuna fisheries. Reviews in Fish Biology and
Fisheries 27, 881--908. \url{https://doi.org/10.1007/s11160-017-9478-x}

\bibitem[\citeproctext]{ref-hartigDHARMaResidualDiagnostics2022}
Hartig, F., 2022.
\href{http://florianhartig.github.io/DHARMa/}{{DHARMa}: {Residual
Diagnostics} for {Hierarchical} ({Multi-Level} / {Mixed}) {Regression
Models}}.

\bibitem[\citeproctext]{ref-kristensenTMBAutomaticDifferentiation2016}
Kristensen, K., Nielsen, A., Berg, C.W., Skaug, H., Bell, B.M., 2016.
{TMB}: {Automatic Differentiation} and {Laplace Approximation}. Journal
of Statistical Software 70. \url{https://doi.org/10.18637/jss.v070.i05}

\bibitem[\citeproctext]{ref-lindgrenExplicitLinkGaussian2011}
Lindgren, F., Rue, H., Lindström, J., 2011. An {Explicit Link} between
{Gaussian Fields} and {Gaussian Markov Random Fields}: {The Stochastic
Partial Differential Equation Approach}. Journal of the Royal
Statistical Society Series B: Statistical Methodology 73, 423--498.
\url{https://doi.org/10.1111/j.1467-9868.2011.00777.x}

\bibitem[\citeproctext]{ref-liuPreliminaryEstimatesBlue2008}
Liu, K.-M., Tsai, W.-P., Joung, S.-J., 2008. Preliminary estimates of
blue and mako sharks bycatch and cpue of {Taiwanese} longline fishery in
the {Atlantic Ocean} (No. SCRS/2008/153). ICCAT (International
Commission for the Conservation of Atlantic Tunas), Madrid, Spain.

\bibitem[\citeproctext]{ref-maternSpatialVariation1986}
Matérn, B., 1986. Spatial {Variation}, 2nd ed. Springer-Verlag, New
York, NY.

\bibitem[\citeproctext]{ref-ortizAlternativeErrorDistribution2004}
Ortiz, M., Arocha, F., 2004. Alternative error distribution models for
standardization of catch rates of non-target species from a pelagic
longline fishery: Billfish species in the {Venezuelan} tuna longline
fishery. Fisheries Research 70, 275--297.
\url{https://doi.org/10.1016/j.fishres.2004.08.028}

\bibitem[\citeproctext]{ref-peatmanEstimatingTrendsMagnitudes2023}
Peatman, T., Allain, V., Bell, L., Muller, B., Panizza, A., Phillip,
N.B., Pilling, G., Nicol, S., 2023. Estimating trends and magnitudes of
bycatch in the tuna fisheries of the {Western} and {Central Pacific
Ocean}. Fish and Fisheries 24, 812--828.
\url{https://doi.org/10.1111/faf.12771}

\bibitem[\citeproctext]{ref-rueApproximateBayesianInference2009}
Rue, H., Martino, S., Chopin, N., 2009. Approximate {Bayesian Inference}
for {Latent Gaussian} models by using {Integrated Nested Laplace
Approximations}. Journal of the Royal Statistical Society Series B:
Statistical Methodology 71, 319--392.
\url{https://doi.org/10.1111/j.1467-9868.2008.00700.x}

\bibitem[\citeproctext]{ref-shonoApplicationTweedieDistribution2008}
Shono, H., 2008. Application of the {Tweedie} distribution to zero-catch
data in {CPUE} analysis. Fisheries Research 93, 154--162.
\url{https://doi.org/10.1016/j.fishres.2008.03.006}

\bibitem[\citeproctext]{ref-stockUtilitySpatialModelbased2019}
Stock, B.C., Ward, E.J., Thorson, J.T., Jannot, J.E., Semmens, B.X.,
2019. The utility of spatial model-based estimators of unobserved
bycatch. ICES Journal of Marine Science 76, 255--267.
\url{https://doi.org/10.1093/icesjms/fsy153}

\bibitem[\citeproctext]{ref-rcoreteamLanguageEnvironmentStatistical2025}
Team, R.C., 2025. \href{https://www.R-project.org/}{R: {A Language} and
{Environment} for {Statistical Computing}}. R Foundation for Statistical
Computing, Vienna, Austria.

\bibitem[\citeproctext]{ref-thorsonGuidanceDecisionsUsing2019}
Thorson, J.T., 2019. Guidance for decisions using the {Vector
Autoregressive Spatio-Temporal} ({VAST}) package in stock, ecosystem,
habitat and climate assessments. Fisheries Research 210, 143--161.
\url{https://doi.org/10.1016/j.fishres.2018.10.013}

\bibitem[\citeproctext]{ref-thorsonModelbasedInferenceEstimating2016}
Thorson, J.T., Pinsky, M.L., Ward, E.J., 2016. Model-based inference for
estimating shifts in species distribution, area occupied and centre of
gravity. Methods in Ecology and Evolution 7, 990--1002.
\url{https://doi.org/10.1111/2041-210X.12567}

\bibitem[\citeproctext]{ref-thorsonGeostatisticalDeltageneralizedLinear2015}
Thorson, J.T., Shelton, A.O., Ward, E.J., Skaug, H.J., 2015.
Geostatistical delta-generalized linear mixed models improve precision
for estimated abundance indices for {West Coast} groundfishes. ICES
Journal of Marine Science 72, 1297--1310.
\url{https://doi.org/10.1093/icesjms/fsu243}

\bibitem[\citeproctext]{ref-thorsonAccountingSpaceTime2013}
Thorson, J.T., Ward, E.J., 2013. Accounting for space--time interactions
in index standardization models. Fisheries Research 147, 426--433.
\url{https://doi.org/10.1016/j.fishres.2013.03.012}

\bibitem[\citeproctext]{ref-tweedieIndexWhichDistinguishes1984}
Tweedie, M.C.K., 1984. An {Index Which Distinguishes} between {Some
Important Exponential Families}, in: Statistics: {Applications} and {New
Directions}: {Proceedings} of the {Indian Statistical Institute Golden
Jubilee International Conference}. Indian Statistical Institute,
Calcutta, Calcutta, pp. 579--604.

\bibitem[\citeproctext]{ref-yanSpatiotemporalModelingBycatch2022}
Yan, Y., Cantoni, E., Field, C., Treble, M., Flemming, J.M., 2022.
Spatiotemporal modeling of bycatch data: Methods and a practical guide
through a case study in a {Canadian Arctic} fishery. Canadian Journal of
Fisheries and Aquatic Sciences 79, 148--158.
\url{https://doi.org/10.1139/cjfas-2020-0267}

\end{CSLReferences}

\newpage{}

\section{Tables}\label{tables}

\begin{longtable}[]{@{}
  >{\raggedright\arraybackslash}p{(\linewidth - 8\tabcolsep) * \real{0.2000}}
  >{\raggedright\arraybackslash}p{(\linewidth - 8\tabcolsep) * \real{0.2000}}
  >{\raggedright\arraybackslash}p{(\linewidth - 8\tabcolsep) * \real{0.2000}}
  >{\raggedright\arraybackslash}p{(\linewidth - 8\tabcolsep) * \real{0.2000}}
  >{\raggedright\arraybackslash}p{(\linewidth - 8\tabcolsep) * \real{0.2000}}@{}}
\caption{List of bycatch taxa by set type (FOB=floating object, FSC=free
school), and their classification (`Group' column).}\tabularnewline
\toprule\noalign{}
\begin{minipage}[b]{\linewidth}\raggedright
Set type
\end{minipage} & \begin{minipage}[b]{\linewidth}\raggedright
Taxon
\end{minipage} & \begin{minipage}[b]{\linewidth}\raggedright
Short name
\end{minipage} & \begin{minipage}[b]{\linewidth}\raggedright
Description
\end{minipage} & \begin{minipage}[b]{\linewidth}\raggedright
Group
\end{minipage} \\
\midrule\noalign{}
\endfirsthead
\toprule\noalign{}
\begin{minipage}[b]{\linewidth}\raggedright
Set type
\end{minipage} & \begin{minipage}[b]{\linewidth}\raggedright
Taxon
\end{minipage} & \begin{minipage}[b]{\linewidth}\raggedright
Short name
\end{minipage} & \begin{minipage}[b]{\linewidth}\raggedright
Description
\end{minipage} & \begin{minipage}[b]{\linewidth}\raggedright
Group
\end{minipage} \\
\midrule\noalign{}
\endhead
\bottomrule\noalign{}
\endlastfoot
FOB & \emph{Elagatis bipinnulata} & E. bipinnulata & - & Common \\
FOB & Balistidae & Balistidae & Mostly \emph{Canthidermis maculata} &
Common \\
FOB & Coryphaenidae & Coryphaenidae & Mostly \emph{Coryphaena hippurus}
& Common \\
FOB & \emph{Acanthocybium solandri} & A. solandri & - & Common \\
FOB & Carangidae & Carangidae & Mostly \emph{Caranx crysos} & Common \\
FOB & Carcharhinidae & Carcharhinidae & Mostly \emph{Carcharhinus
falciformis} & Special interest \\
FOB & \emph{Makaira nigricans} & M. nigricans & - & Special interest \\
FOB & Sphyrnidae & Sphyrnidae & Mostly \emph{Sphyrna mokarran},
\emph{Sphyrna lewini}, and \emph{Sphyrna zygaena} & Special interest \\
FOB & Cheloniidae & Cheloniidae & Mostly \emph{Eretmochelys imbricata},
\emph{Chelonia mydas}, \emph{Lepidochelys olivacea}, \emph{Lepidochelys
kempii}, and \emph{Dermochelys coriacea} & Special interest \\
FOB & Mobulidae & Mobulidae & Mostly \emph{Mobula birostris} and
\emph{Mobula mobular} & Special interest \\
FOB & Alopiidae & Alopiidae & Mostly \emph{Alopias vulpinus} & Rare \\
FOB & \emph{Xiphias gladius} & X. gladius & - & Rare \\
FOB & Lamnidae & Lamnidae & Mostly \emph{Lamna nasus} and \emph{Isurus
oxyrinchus} & Rare \\
FOB & \emph{Prionace glauca} & P. glauca & - & Rare \\
FSC & Carcharhinidae & Carcharhinidae & See above & Common \\
FSC & Mobulidae & Mobulidae & See above & Common \\
FSC & \emph{Istiophorus albicans} & I. albicans & - & Common \\
FSC & \emph{Makaira nigricans} & M. nigricans & - & Special interest \\
FSC & Sphyrnidae & Sphyrnidae & See above & Special interest \\
FSC & Cheloniidae & Cheloniidae & See above & Special interest \\
FSC & Molidae & Molidae & Mostly \emph{Mola mola} & Special interest \\
FSC & Lamnidae & Lamnidae & See above & Rare \\
FSC & \emph{Prionace glauca} & P. glauca & - & Rare \\
FSC & Istiophoridae & Istiophoridae & Other marlin species than
\emph{Istiophorus albicans} and \emph{Makaira nigricans} & Rare \\
\end{longtable}

\newpage{}

\section{Figures}\label{figures}




\end{document}
